\documentclass[12pt,letterpaper,]{book}
\usepackage{lmodern}
\usepackage{amssymb,amsmath}
\usepackage{ifxetex,ifluatex}
\usepackage{fixltx2e} % provides \textsubscript
\ifnum 0\ifxetex 1\fi\ifluatex 1\fi=0 % if pdftex
  \usepackage[T1]{fontenc}
  \usepackage[utf8]{inputenc}
\else % if luatex or xelatex
  \ifxetex
    \usepackage{mathspec}
  \else
    \usepackage{fontspec}
  \fi
  \defaultfontfeatures{Ligatures=TeX,Scale=MatchLowercase}
    \setmainfont[]{Cambria}
    \setsansfont[]{Arial}
    \setmathfont(Digits,Latin,Greek)[]{Cambria}
\fi
% use upquote if available, for straight quotes in verbatim environments
\IfFileExists{upquote.sty}{\usepackage{upquote}}{}
% use microtype if available
\IfFileExists{microtype.sty}{%
\usepackage{microtype}
\UseMicrotypeSet[protrusion]{basicmath} % disable protrusion for tt fonts
}{}
\usepackage[top=2.5cm, bottom=2.5cm, left=3cm, right=3cm]{geometry}
\usepackage{hyperref}
\hypersetup{unicode=true,
            pdftitle={Ecología de poblaciones silvestres},
            pdfauthor={David Martínez Cascante},
            pdfborder={0 0 0},
            breaklinks=true}
\urlstyle{same}  % don't use monospace font for urls
\usepackage{natbib}
\bibliographystyle{DAVID}
\usepackage{color}
\usepackage{fancyvrb}
\newcommand{\VerbBar}{|}
\newcommand{\VERB}{\Verb[commandchars=\\\{\}]}
\DefineVerbatimEnvironment{Highlighting}{Verbatim}{commandchars=\\\{\}}
% Add ',fontsize=\small' for more characters per line
\usepackage{framed}
\definecolor{shadecolor}{RGB}{248,248,248}
\newenvironment{Shaded}{\begin{snugshade}}{\end{snugshade}}
\newcommand{\KeywordTok}[1]{\textcolor[rgb]{0.13,0.29,0.53}{\textbf{#1}}}
\newcommand{\DataTypeTok}[1]{\textcolor[rgb]{0.13,0.29,0.53}{#1}}
\newcommand{\DecValTok}[1]{\textcolor[rgb]{0.00,0.00,0.81}{#1}}
\newcommand{\BaseNTok}[1]{\textcolor[rgb]{0.00,0.00,0.81}{#1}}
\newcommand{\FloatTok}[1]{\textcolor[rgb]{0.00,0.00,0.81}{#1}}
\newcommand{\ConstantTok}[1]{\textcolor[rgb]{0.00,0.00,0.00}{#1}}
\newcommand{\CharTok}[1]{\textcolor[rgb]{0.31,0.60,0.02}{#1}}
\newcommand{\SpecialCharTok}[1]{\textcolor[rgb]{0.00,0.00,0.00}{#1}}
\newcommand{\StringTok}[1]{\textcolor[rgb]{0.31,0.60,0.02}{#1}}
\newcommand{\VerbatimStringTok}[1]{\textcolor[rgb]{0.31,0.60,0.02}{#1}}
\newcommand{\SpecialStringTok}[1]{\textcolor[rgb]{0.31,0.60,0.02}{#1}}
\newcommand{\ImportTok}[1]{#1}
\newcommand{\CommentTok}[1]{\textcolor[rgb]{0.56,0.35,0.01}{\textit{#1}}}
\newcommand{\DocumentationTok}[1]{\textcolor[rgb]{0.56,0.35,0.01}{\textbf{\textit{#1}}}}
\newcommand{\AnnotationTok}[1]{\textcolor[rgb]{0.56,0.35,0.01}{\textbf{\textit{#1}}}}
\newcommand{\CommentVarTok}[1]{\textcolor[rgb]{0.56,0.35,0.01}{\textbf{\textit{#1}}}}
\newcommand{\OtherTok}[1]{\textcolor[rgb]{0.56,0.35,0.01}{#1}}
\newcommand{\FunctionTok}[1]{\textcolor[rgb]{0.00,0.00,0.00}{#1}}
\newcommand{\VariableTok}[1]{\textcolor[rgb]{0.00,0.00,0.00}{#1}}
\newcommand{\ControlFlowTok}[1]{\textcolor[rgb]{0.13,0.29,0.53}{\textbf{#1}}}
\newcommand{\OperatorTok}[1]{\textcolor[rgb]{0.81,0.36,0.00}{\textbf{#1}}}
\newcommand{\BuiltInTok}[1]{#1}
\newcommand{\ExtensionTok}[1]{#1}
\newcommand{\PreprocessorTok}[1]{\textcolor[rgb]{0.56,0.35,0.01}{\textit{#1}}}
\newcommand{\AttributeTok}[1]{\textcolor[rgb]{0.77,0.63,0.00}{#1}}
\newcommand{\RegionMarkerTok}[1]{#1}
\newcommand{\InformationTok}[1]{\textcolor[rgb]{0.56,0.35,0.01}{\textbf{\textit{#1}}}}
\newcommand{\WarningTok}[1]{\textcolor[rgb]{0.56,0.35,0.01}{\textbf{\textit{#1}}}}
\newcommand{\AlertTok}[1]{\textcolor[rgb]{0.94,0.16,0.16}{#1}}
\newcommand{\ErrorTok}[1]{\textcolor[rgb]{0.64,0.00,0.00}{\textbf{#1}}}
\newcommand{\NormalTok}[1]{#1}
\usepackage{longtable,booktabs}
\usepackage{graphicx,grffile}
\makeatletter
\def\maxwidth{\ifdim\Gin@nat@width>\linewidth\linewidth\else\Gin@nat@width\fi}
\def\maxheight{\ifdim\Gin@nat@height>\textheight\textheight\else\Gin@nat@height\fi}
\makeatother
% Scale images if necessary, so that they will not overflow the page
% margins by default, and it is still possible to overwrite the defaults
% using explicit options in \includegraphics[width, height, ...]{}
\setkeys{Gin}{width=\maxwidth,height=\maxheight,keepaspectratio}
\IfFileExists{parskip.sty}{%
\usepackage{parskip}
}{% else
\setlength{\parindent}{0pt}
\setlength{\parskip}{6pt plus 2pt minus 1pt}
}
\setlength{\emergencystretch}{3em}  % prevent overfull lines
\providecommand{\tightlist}{%
  \setlength{\itemsep}{0pt}\setlength{\parskip}{0pt}}
\setcounter{secnumdepth}{5}
% Redefines (sub)paragraphs to behave more like sections
\ifx\paragraph\undefined\else
\let\oldparagraph\paragraph
\renewcommand{\paragraph}[1]{\oldparagraph{#1}\mbox{}}
\fi
\ifx\subparagraph\undefined\else
\let\oldsubparagraph\subparagraph
\renewcommand{\subparagraph}[1]{\oldsubparagraph{#1}\mbox{}}
\fi

%%% Use protect on footnotes to avoid problems with footnotes in titles
\let\rmarkdownfootnote\footnote%
\def\footnote{\protect\rmarkdownfootnote}

%%% Change title format to be more compact
\usepackage{titling}

% Create subtitle command for use in maketitle
\newcommand{\subtitle}[1]{
  \posttitle{
    \begin{center}\large#1\end{center}
    }
}

\setlength{\droptitle}{-2em}
  \title{Ecología de poblaciones silvestres}
  \pretitle{\vspace{\droptitle}\centering\huge}
  \posttitle{\par}
  \author{David Martínez Cascante}
  \preauthor{\centering\large\emph}
  \postauthor{\par}
  \predate{\centering\large\emph}
  \postdate{\par}
  \date{2018-02-21}

\usepackage[spanish,es-nodecimaldot]{babel}
\usepackage{booktabs}
\usepackage{makeidx}
\makeindex
\usepackage{ragged2e}
\usepackage{cancel}
\usepackage{placeins}
\usepackage{siunitx}
\sisetup{detect-all = true, detect-family=true} 
\usepackage{setspace}
\usepackage{chngcntr}
\counterwithin{figure}{section}
\counterwithin{table}{section}
\onehalfspacing
\newtheorem{theorem}{Teorema}
\newtheorem{algorithm}{Algoritmo}
\newtheorem{axiom}{Axioma}
\newtheorem{definition}{Definición}
\newtheorem{example}{Ejemplo}
\newtheorem{exercise}{Ejercicio}
\newtheorem{lemma}{Lemma}
\newtheorem{proposition}{Proposición}
\newtheorem{remark}{Remarca}
\newtheorem{solution}{Solución\;\thesection\,.}
\newtheorem{summary}{Resumen}
\usepackage{fancyhdr}
\pagestyle{fancy}
\lhead{ECB }
\rhead{Ecología de Poblaciones}
\RaggedRight
\setlength\parindent{24pt}
\usepackage{amssymb}

\let\BeginKnitrBlock\begin \let\EndKnitrBlock\end
\begin{document}
\maketitle

{
\setcounter{tocdepth}{1}
\tableofcontents
}
\chapter{Introducción}\label{intro}

\chapter{Modelos de crecimiento}\label{modelos-de-crecimiento}

\section{Crecimiento
denso-independiente}\label{crecimiento-denso-independiente}

\subsection{Crecimiento geométrico}\label{crecimiento-geometrico}

\subsubsection{Ejemplos}\label{ejemplos}

\subsubsection{Ejercicios}\label{ejercicios}

\subsection{Crecimiento exponencial}\label{crecimiento-exponencial}

\subsubsection{Ejemplos}\label{ejemplos-1}

\subsubsection{Ejercicios}\label{ejercicios-1}

\section{Crecimiento
denso-dependiente}\label{crecimiento-denso-dependiente}

\subsection{Matrices}\label{matrices}

\section{Otras fuentes
bibliográficas}\label{otras-fuentes-bibliograficas}

\chapter{Soluciones a los ejercicios}\label{soluciones-a-los-ejercicios}

\appendix


\chapter{Placeholder}\label{placeholder}

\chapter{Métodos numéricos para ecología de
poblaciones}\label{metodos-numericos-para-ecologia-de-poblaciones}

\section{Simulación de ecuaciones
diferenciales}\label{simulacion-de-ecuaciones-diferenciales}

\chapter{Tutorial de R con RStudio}\label{tutorial-de-r-con-rstudio}

\section{\texorpdfstring{Crear un proyecto en
\emph{RStudio}}{Crear un proyecto en RStudio}}\label{RStudioProject}

\section{Funciones básicas en R}\label{funciones-basicas-en-r}

\section{Estructuras de datos}\label{estructuras-de-datos}

\section{Funciones}\label{funciones}

\chapter{Asignaciones}\label{asignaciones}

\section{\texorpdfstring{Tarea 01: ¡Hola mundo con
\emph{Rmarkdown}!}{Tarea 01: ¡Hola mundo con Rmarkdown!}}\label{tarea-01-hola-mundo-con-rmarkdown}

\textbf{Objetivo}: Verificar que el estudiante ha instalado, y maneja el
ambiente de trabajo que se utilizará durante el curso.

Primero revisa los enlaces provistos en el
\href{https://github.com/dawidh15/dinPob/wiki/02-Instalaci\%C3\%B3n-del-software-necesario\#prueba-con-rmarkdown}{wiki}.

\textbf{Actividades}

\begin{itemize}
\item
  Haz un nuevo proyecto en \textbf{RStudio}, que se llame
  \emph{Tarea01}. Ver pasos en sección \ref{RStudioProject}.
\item
  En la consola de \textbf{R}, escribe
  \texttt{install.packages(rmarkdown)}, con todas las dependencias. O
  instala el paquete desde \textbf{RStudio} como se mostró en el
  \href{https://github.com/dawidh15/dinPob/wiki/02-Instalaci\%C3\%B3n-del-software-necesario}{wiki}.
\item
  En \textbf{RStudio}
  \texttt{File-\/-\textgreater{}\ New\ File\ -\/-\textgreater{}\ R\ Markdown}.
\item
  Crea una sección principal que se llame \emph{Información
  profesional}.
\item
  Luego, crea una sección secundaria que se llame \emph{Intereses}. Usa
  bullets para nombrar algunos intereses profesionales.
\item
  Luego, crea una sección secundaria llamada \emph{Experiencia Laboral},
  si aplica. Nombra algunos trabajos relacionados con el curso de
  Ecología de Poblaciones.
\item
  Crea una sección principal que se llame \emph{Integración con R}
\item
  Consigue algunos datos interesantes en internet. Deben ser datos para
  graficar, por tanto deben tener dos columnas, y varias filas. Puedes
  ir a \href{https://www.wolframalpha.com/}{Wolfram Alpha}. Guarda los
  datos como un texto delimitado por comas (\texttt{.csv}).
\item
  En \textbf{R} o \textbf{RStudio} corre el comando
  \texttt{?read.table}. Para correr un comando en \textbf{RStudio}
  apreta \texttt{Cntrl\ +\ R}.
\item
  Crea un ``\emph{chunk}'' de código. Esto se hace en \textbf{RStudio},
  busca un botón en la barra especial de \emph{rmarkdown} que diga
  \texttt{insert}, luego escoge \texttt{R}.
\item
  Lee la tabla y asignala a un objeto:
\end{itemize}

\begin{Shaded}
\begin{Highlighting}[]
\NormalTok{datos <-}\StringTok{ }\KeywordTok{read.table}\NormalTok{(}\OperatorTok{<}\NormalTok{ruta_de_archivo_en_comillas}\OperatorTok{>}\NormalTok{,}
                    \DataTypeTok{header =} \OtherTok{TRUE}\NormalTok{,}
                    \DataTypeTok{sep =} \StringTok{","}\NormalTok{)}
\end{Highlighting}
\end{Shaded}

\begin{itemize}
\item
  Grafica los datos en un nuevo ``\emph{chunk}''. Usa el método que
  prefieras. Hay mucho material de cómo hacer gráficos en R. Por ahora,
  un gráfico básico es suficiente.
\item
  Ahora, haz otra sección llamada \emph{Bibliografía}. En un párrafo
  escribe una mini-revisión de algún tema que domines y del que
  dispongas referencias bibliográficas. Usa los mecanismos de citas de
  \emph{rmarkdown}

  \begin{itemize}
  \item
    Cita en texto con \texttt{@citationKey}
  \item
    Cita en paréntesis con \texttt{{[}@citationKey{]}}
  \end{itemize}
\end{itemize}

\begin{center}\rule{0.5\linewidth}{\linethickness}\end{center}

\textbf{Importante}: Para que las citas funcionen, debes agregar unas
opciones en la \emph{cabecera} del documento (\emph{YAML header}):

\begin{verbatim}
bibliography: <tu_archivo_bib>.bib
csl: apa.csl
\end{verbatim}

El archivo \texttt{apa.csl} se puede encontrar en google. Es un archivo
de estilo APA, para dar formato a la bibliografía. Revisa
\href{https://www.zotero.org/styles}{el repositorio de CSL de Zotero},
en busca de las revistas disponibles.

\begin{center}\rule{0.5\linewidth}{\linethickness}\end{center}

\begin{itemize}
\tightlist
\item
  Por último, corre el documento con el botón \texttt{knit}. Envía el
  documento \texttt{.Rmd} y el \texttt{.pdf} al profesor
  (\texttt{dawidh15@gmail.com}).
\end{itemize}

\newpage

\section{Tarea 02: Ejercicios de
crecimiento}\label{tarea-02-ejercicios-de-crecimiento}

Se recomienda hacer primero todo en papel, y luego pasarlo en limpio
usando \emph{Rmarkdown}. LINKS para LatexMath

\setcounter{exercise}{0}

\BeginKnitrBlock{exercise}
\protect\hypertarget{exr:T02E01}{}{\label{exr:T02E01} }Resuelva el siguiente
ejercicio de crecimiento exponencial
\EndKnitrBlock{exercise}

En un laboratorio se cultiva una especie presa para un programa de
reintroducción de una especie de pez. En el laboratorio, se inició un
proyecto de mejora en la producción de la presa, y se ha diseñado un
experimento para aumentar el valor nutricional de las presas.

Se cuenta con un presupuesto de \SI{2e6}{\text{CRC}} para la producción
de animales presa en el proyecto. Además, el diseño experimental
requiere de 40 recipientes acondicionados con diferentes tratamientos.
Las presas crecen con una tasa de crecimiento intrínseco de \num{0.098}.
Además, el inóculo inicial es de \num{1000} individuos por recipiente.
Si se sabe que el costo de mantenimiento por organismo-día es de
\SI{0.5}{\text{CRC}\per(\text{ind}.\day)}:

\emph{¿Cuántos organismos por recipiente se pueden cultivar sin
sobrepasar el dinero disponible? ¿Cuánto tiempo, en días, se necesitan
para alcanzar esa cantidad?}

\emph{Tips}:

\begin{itemize}
\item
  Este es un problema de mínimos. Primero hay que buscar la función a
  minimizar. Luego, uno encuentra el valor apropiado del parámetro de
  interés cuando la función se minimiza.
\item
  Para minimizar una diferencia, use el valor absoluto de la diferencia:
  \(\left| x - y\right|\).
\item
  Use la función \texttt{optim} o \texttt{optimize}. Una vez que tenga
  la función que desea minimizar escrita en \emph{R} use este código:
\end{itemize}

\begin{verbatim}
out <- optim(par = 0,fn = <nombre_de_funcion_para_minimizar>,
      control = list(reltol=0.01),
      method = "Brent", 
      lower = <numero>, 
      upper = <numero>)
\end{verbatim}

\begin{itemize}
\tightlist
\item
  Antes se recomienda buscar la ayuda de la función en la consola de
  \textbf{R}, al escribir \texttt{?optim}. Revise los ejemplos, y lea
  detalladamente la ayuda.
\end{itemize}

\bibliography{book.bib,packages.bib}


\end{document}
